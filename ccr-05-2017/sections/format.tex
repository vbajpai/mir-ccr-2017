%**************************************************************************
\section{Retreat Format} \label{sec:intro}
%**************************************************************************

The main concern while organizing the retreats is to create an opportunity where researchers of academia and industry meet each other and discuss about their work. We try to put this goal into practice with the approach described below. Some selected elements were adapted from seminars held in the renowned Leibniz Center for Informatics in Dagstuhl, Germany~\cite{dagstuhl}.

1) We hold the MIRR in the \ac{TUM} Science and Study Center in Raitenhaslach, a former monastery, away from the daily activities of our participants to ensure focus. We also include an overnight stay and a social dinner to foster continued interaction and allow for digesting ideas.
2) The seminar is by invitation only. We put an emphasis on the industry and hand pick PhD students or post-docs with matching topics. This helps with obtaining a compatible and energetic mix of people.
3) We limit the number of participants to $\sim$40 to maintain interactivity and allow all participants to meet one another and create professional contacts.
4) MIRR is organized to provide a maximum of interactive action points on the agenda.
Each participant is requested to bring two slides that include his/her photo, some keywords describing ones current research focus, and questions that he/she likes to discuss during the retreat. The slides are used at the beginning of the retreat in a self-introduction session to familiarize participants with another.
In the first two iterations of \ac{MIRR}, we already limited ourselves to about six 20 minutes talks which allows us to dedicate most of the time to more interactive formats like poster or breakout sessions.
We furthermore ask all participants to bring a poster. Posters provide variety of topics and introduce the participant's research more profoundly than the short intro slides, and in a more personal and interactive fashion than talks.
Furthermore we emphasize discussions both in the plenum as well as in small subgroups (breakout sessions). For breakouts, the plenum first agrees on most relevant topics or questions previously presented by individuals and forms groups with sizes between three to a maximum of six persons. The discussions of the various groups and their most interesting insights are later shared with the plenum in five to ten minutes, improvised talks.
For our tentative agenda, see table \ref{tbl:agenda}.
5) Because we know that everybody's time is scarce, we organize each retreat in a way that it occupies just two days including arrival and departure. With a target of two workshops per year, presently scheduled for May and November, we shall be able to continuously engage with a growing regional community even if individuals cannot participate on every occasion.
6) As we previously mentioned, organization directions are shaped by the feedback of the participants, keeping the format constantly improving. 

\begin{table}[]
\centering
\caption{Tentative MIRR Agenda}
\label{tbl:agenda}
\begin{tabular}{ l l }
\multicolumn{2}{c}{\textbf{Day 1}} \\ \hline
10:00 &
Welcome and self-introduction of participants \\
12:00 &
Lunch\\
13:00 &
Poster session 1\\
14:00 &
Talk session 1\\
16:00 &
Poster session 2\\
17:00 &
Breakout sessions\\
18:30 &
Social dinner\\
\\
\multicolumn{2}{c}{\textbf{Day 2}} \\ \hline
09:00 &
Reports from breakout sessions\\
11:00 &
Talk session 2\\
13:00 &
Lunch\\
14:00 &
Closing discussion and evolving the retreat\\
15:00 &
End\\

\end{tabular}
\end{table}
