\section{Retreat Format}

The main concern while organizing the retreats is to create an opportunity where researchers meet each other and discuss about their work. We try to put this goal into practice with the  approach described below. Some selected elements were adapted from seminars held in the renowed Leibniz Center for Informatics in Dagstuhl\cite{dagstuhl}.

1) We hold the MIRR in the \ac{TUM} Science and Study Center in Raitenhaslach, a former monastry, away from the daily activities of our participants to ensure focus. We also include an overnight stay and a social dinner to foster continued interaction and allow for digesting ideas.
2) The seminar is by invitation only. We put an emphasis on the industry and hand pick PhD students or post-docs with matching topics. This helps with obtaining a compatible and energetic mix of people.
3) We limit the number of participants to $\sim$40 to maintain interactivity and allow all
participants to meet one another.
4) MIRR is organized in a highly interactive fashion. So far we limited ourselfs to about six 20 minutes talks which allows us to dedicate most of the time to more interactive formats. Each participant is requested to bring two slides that include his/her photo, current research focus, and questions that he/she likes to discuss during the retreat. Furthermore, participants are asked to bring a poster. Posters provide variety and introduce the participant's research in a more personal and interactive manner than talks would achieve. Furthermore we emphasize discussions both in the plenum as well as in small subgroups (breakout sessions). For breakouts, participants first agree on most relevant topics or questions brought and form groups with sizes between three to max. eight persons. Outcomes and results of the breakout sessions are later shared and discussed with the plenum. For our tentaive agenda, see table \ref{tbl:agenda}
4) Because we know that everybody's time is scarce, we organize each retreat in a way that it occupies just two days including arrival and departure.  With a target of two workshops per year, presently scheduled for May and November, we shall be able to continuously
engage with a growing regional community even if individuals cannot
participate on every occasion.
5) Organization directions are shaped by the feedback of the participants, keeping the format constantly improving. We present the feedback of participants in section \ref{sec:evolution}.


\begin{table}[]
\centering
\caption{Tentative MIRR Agenda}
\label{tbl:agenda}
\begin{tabular}{ l l }
\multicolumn{2}{c}{\textbf{Day 1}} \\ \hline
10:00 &
Welcome and self-introduction of participants \\
12:00 &
Lunch\\
13:00 &
Poster session 1\\
14:00 &
Talk session 1\\
16:00 &
Poster session 2\\
17:00 &
Breakout sessions\\
18:30 &
Social dinner\\
\\
\multicolumn{2}{c}{\textbf{Day 2}} \\ \hline
09:00 &
Reports from breakout sessions\\
11:00 &
Talk session 2\\
13:00 &
Lunch\\
14:00 &
Closing discussion and evolving the retreat\\
15:00 &
End\\

\end{tabular}
\end{table}
