%**************************************************************************
\section{Introduction}\label{sec:introduction}
%**************************************************************************

%------------------------ Motivation
%------------------------ Goals

The \ac{MIRR} originated from informal discussions of different research groups at \ac{TUM} and a team at the Munich branch of NetApp\cite{netapp} on diverse topics related to networking.
In these meetings, PhD students and postdoctoral fellows (post-docs) presented their respective research, including both work in progress as well as polished results.
The meetings created an informal setting for intense and rich exchange among participants.
We realised that there was notable potential in reaching out further, which eventually led to the instantiation of the \ac{MIRR}.

The main mission of the \ac{MIRR} is to ensure mutual awareness of different
teams working on current (complementary) topics in networking. Our scope ranges from network measurements, to systems engineering, to security and privacy problems in networks.
We want to lay
the foundations for establishing, broadening, and deepening cooperation among
a variety of groups doing networking research. In order to foster easily
sustainable relationships, our initial scope has been deliberately limited to
the area around Munich (which may reach as far as 400 km in some cases).  As a
common denominator, we target like-minded teams within the region, where the
common mindset stems from practical research in networked systems, paired with
interest and efforts in the \ac{IETF}, the \ac{IRTF} and the ACM SIGCOMM and
SIGMOBILE communities.

The purpose of the \ac{MIRR} is threefold: 1) We seek to provide recurring
opportunities for companies to get in touch with research groups that have
expertise in fields relevant to the former.  2) We aim to support researchers
in understanding current and emerging research and engineering problems from
the commercial development and deployment perspectives.  3) We like to offer
reality feedback to academic researchers and out-of-the-box ideas to those
from industry.  Overall, we hope to foster future bi- or multi-lateral
collaboration between academics and industry.

Towards this mission, the 1\textsuperscript{st} \ac{MIRR} retreat was organized
on November 24--25, 2016 at the \ac{TUM} Science and Study Center in
Raitenhaslach, Germany\cite{raitenhaslach}. A 2\textsuperscript{nd} iteration of the \ac{MIRR} was
organized at the same location and held on May 23--24, 2017.
