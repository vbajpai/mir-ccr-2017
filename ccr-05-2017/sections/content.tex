%**************************************************************************
\section{Overview on Selected Topics} \label{sec:content}
%**************************************************************************

In this section we provide an overview on the topics that have been
covered in invited talks, breakout sessions, and posters through MIRR.

\subsection{System Networking and Measurement}

A major theme covered in MIRR is on networking and measurement, which
encompasses topics of Software-Defined Networking (SDN), Network Function
Virtualization (NFV), Internet architecture, networking flexibility, 
reproducibility, BGP and MultiPath routing management, IPv6 measurement,  
protocol development of QUIC, and network stack optimization. 

%Talks, place holder

In the talk ``On software Network Management" by Artur Hecker (Huawei), 
he argued that
the paradigm change brought by software networks does not suit well planning
approaches for network dimensioning and design, including but not limited to
planning or pre-provisioning of management and control planes. In contrast, he
proposed a new model and protocol, aiming to autonomously
bootstrap, construct, adjust and maintain control plane including the elastic
placement of control compute nodes ~\cite{yliu:icc:2015} and control paths without
presuming any particular network purpose.

The talk by Wolfgang Kellerer (TUM) on ``FlexNets: Quantifying Flexibility in
Communication Networks" proposed a definition for flexibility as a
new measure for network design space analysis~\cite{wkellerer:infocom:2016}
and gives an illustrative example with \ac{SDN} controller placement.
The goal is to provide a deeper understanding of the
flexibility vs.\ cost trade-off which is missing so far in networking research.

Brian Trammell (ETH Zürich) introduced PostSockets \cite{draft-trammell-post-sockets}
in his talk on ``An Accidental Internet Architecture". The work-in-progress 
proposal intended to allow applications to be developed separate from (possibly
runtime-bound) transport protocol dynamics, in turn accelerating the
deployment of recent innovations at Layer 4.

The talk on ``Measuring IPv6 performance", Vaibhav Bajpai (Jacobs University
Bremen) presented his work using 80
dual-stacked SamKnows~\cite{vbajpai:comst:2015} probes deployed at the edge of
the network to measure IPv6 performance of operational dual-stacked content
services on the Internet. The talk presented a comparison of how content delivery
\cite{vbajpai:networking:2015, sahsan:pam:2015} over IPv6 compares to that of
IPv4, and also identified glitches in this content
delivery~\cite{seravuchira:cnsm:2016} that once fixed can help improve user
experience over IPv6. The talks also pointed out areas of improvements
\cite{vbajpai:anrw:2016} in the standards work for the IPv6 operations
community at the IETF\@.

%citations should be all done properly now :)

In ``SWIFT: Predictive Fast Reroute upon Remote BGP Disruptions",
Laurent Vanbever (ETH Zurich) presented a general
fast re-route framework supporting both local and remote failures. The framework
SWIFT isbased on two novel techniques. First, SWIFT copes with slow notification by
predicting the overall extent of a remote failure out of few control-plane
(BGP) messages. Second, SWIFT introduces a new data-plane encoding scheme
which enables it to quickly and flexibly update the impacted forwarding
entries. The evaluation has shown that SWIFT is able to predict the extent of
a remote failure with high accuracy (93\%) and SWIFT encoding scheme enables
to fast-converge more than 95\% of the impacted forwarding entries. Overall,
SWIFT reduces the average convergence time from few minutes to few seconds.

Hagen Paul Pfeifer presented in ``Dynamic MultiPath Routing Protocol" his 
proposal \ac{DMPR}, which provides exterior routing protocol features for
heterogeneous link layer environments even in low bandwidth environments.
\ac{DMPR} features policy based routing to route traffic through
different paths if required or advantageous.

In their talks ``Persistence in Networking: Redesigning Stack, API and Networks"
and ``State of Linux Network Development", Michio Honda (NEC) and 
Florian Westphal (Redhat) highlighted the existing status of kernel
development and the demand for more efficient network stack and
its APIs. Their talks illusrated new opportunities in networking such as software
switches, NIC hardware offloading, extension of the BSD socket API to 
provide a full zero-copy interface and improvement of networked
storage systems.

%Group discussion, place holder

Focusing on networking aspects, the discussion group
on ``SDN/NFV Measurements and Applications Perspective"
stressed the importance of performance evaluation for virtualized networks and
SDN networks. As measurements should be conducted on software platforms as well
as real networks, testbeds need to be built that include
commodity servers, making use of accelerated network cards via
\ac{DPDK}, networking functions in software (e.g., functions running in docker
containers, orchestration tools for virtual environments).
Besides system perspective, the discussion group also illustrated the perspective
of how to apply SDN, and how to introduce improvements to SDN 
for better meeting the identified requirements. 


The group on ``QUIC Development" discussed how
the \ac{IETF} should approach the information encrypted in the QUIC
packets in respect to its availability to legitimate network management or firewall
functions. The session also went into retrospect on historical protocol
innovations that failed to get widely deployed and reflect on how to enable
smooth transitions of future protocol innovations. The group also
discussed the role of operator networks by remaining opaque to
designers of network protocols, and additional large-scale
measurement initiatives that help bring visibility into how current network
operate in practice would be useful for protocol innovation.


The discussion group ``Measurements and Reproducibility" pointed out 
a challenging issue in networking community where there is an inbalance between
quickly publishing novel results and providing reproducible
research~\cite{vbajpai:reproducibility:2017, qscheitle:reproducibility:2017}.
The group rised questions regarding user privacy and sharing
sensitive (network traffic, user identities, location, et al.) data since both
legal and ethical concerns often inhibit data sharing, and current state and
possible directions for improving the state of reproducibility.
Besides methodology, the group also discussed what can we learn from other fields
where validation of results through replication has been an essential component.


The group session on ``Networking APIs" focused on the assumption and problems
observed with the current APIs. The group first defined the assumptions
under which the flaws and possible improvements of the Networking APIs, in
particular the standard Socket API, should take place.
This led to a discussion of desirable properties of a networking API.
Mentioned were the isolation of networking-stack and application, energy
efficiency (mobile applications) and high performance and scalability (data
center applications).

%Posters, place holders

Following active discussions, on-going work has been presented in 
several posters, including 
``Cost of Security in the SDN Control Plane", ``The Baltikum Testbed",
``FlexNets", ``Boost Virtual Network Resource Allocation",
``HyperFlex", ``SarDiNe", ``StackMap", ``PATHspider",
``PASTE: A Networking Interface for NVMMs",
``Real-time TE in the Internet",
``Center ofAutomotiveResearch on Integrated Safety Systems
and Measurement Area (CARISSMA)",
``Project SENDATE", and
``Quantifying Flexibility in Networks".

\subsection{IoT, ICN and Edge Computing}

The second theme covered in MIRR is the intersection of IoT, Information-Centric
Networking (ICN) and Edge Computing. There were three invited talks that triggered
lively discussions. 

In the talk ``Edge Computing considered harmful" by Dirk
Kutscher (Huawei), he challenged the mainstream notion of running
application-specific VMs at the network edge and discussed the related
security/privacy issues. He argued that low latency should be first-order
general requirement and will point at corresponding network and
transport layer approaches.

The talk ``Open Platforms for Cyber-physical systems" by Christian Prehofer (fortiss)
illustrated a timely trend towards open systems
which can be extended during operation by instantly adding functionalities on
demand. He presented results of the
TAPPlications (Trusted Applications for open CPS) project~\cite{prehofer:eitec:2016},
which includes trusted hardware and
virtualization of networking and CPU, as well as dedicated execution
environments and development support for trusted applications.

Teemu Kärkkäinen (TU Munich) presented in ``Opportunistic Content
Dissemination in Dense Network Segments" a 50 client opportunistic network in
a single Wi-Fi access point and use it to uncover scaling problems and to suggest
mechanisms to improve the performance of single segment dissemination. He further
proposed an algorithm for breaking down a single dense segment dissemination
problem into multiple smaller but identical problems by exploiting resource 
(e.g., Wi-Fi channel) diversity, and validate our approach via simulations and
testbed experiments.

Inspired by those talks, discussion groups on IoT, ICN and Edge Computing
have been formed to review the existing issues, open problems, and
research directions. The identified problems include: 1) Limitation
of existing protocols such as Constrained Application Protocol (CoAP) that
handles poorly the frequent leaving/joining events in the network. 2) The
stereotype of ``IoT gateway design'' has hindered novel design. 3) We still
have not yet come up with a suitable Internet architecture that integrates
IoT coherently.

There are six open questions highlighted by the group: 1) Where does the network
end nowadays? This question couples
with the ICN where nodes can contribute to the computation/content along
the path. 2) What functions on gateway functions we can remove? 3) How to do
naming ``translation'' without changing name/label? 4) Can we do packet
processing while it is passing through queue? 5) How to avoid looping in
the network functions? This is a key concern since we need to keep a
boundary for resource usage in the network. 6) How to maintain the state on
the constrained nodes?

The discussion groups also identified a set of potential
research directions: 1) Design of end-to-end naming scheme, to facilitate IoT
application composition and bring down the overhead of porting applications
for the cloud to ``gateways''. 2) Semantics for individual sensor and
equivalence group. 3) Trade accuracy with replication. 4) A new computation
abstract suitable for IoT. 5) Abstract of distributed registry for network
function. 6) Rethink how we distribute computing and content.

Besides invited talks, MIRR also featured several posters, including
topics on IoT edge computing ``iConfig - What I See is What I Configure",
``Fine-Grained Edge Offloading for IoT"~\cite{cozzolino:hotconnet:2017},
``Accountability for Cyber-Physical Systems", from the angle of 
mobility and opportunistic content distribution ``Opportunistic Content Dissemination",
``Data-driven Mobility Modeling",
``Data Dissemination in Vehicular Networks", and vehicular edge virtualization
``Lightweight Virtualization for Smart Cars"~\cite{rmorabito:im:2017},
``Car2X Lab".

% some of posters above are missing references - in case that caused the confusion.

\subsection{Security and Privacy}

Topics around security and privacy form the third thematic pillar of MIRR.
Topics spanning the full spectrum from hardware-related, to protocol-specific,
to problems of and introduced by cloud computing were covered.

%Talks

In their talks ``Digital Sovereignty in the Post-Snowden Era'' and ``IoT
Security: TrustZone for v8-M'' Alexander von Gernler (genua GmbH) and Hannes
Tschofenig (ARM) motivated for hardware offering security features implemented
in hardware resp. reported on the latest developments in the
8\textsuperscript{th} version of the ARM A- and M-class processors.

Quirin Scheitle (TUM) presented his investigations on ``User Tracking Based on
TLS Client Certificate Authentication''~\cite{qscheitle:tma:2017}. He explained
that currently devices often transmit their certificates in plain text and
demonstrated the impact of this problem on client traceability using the Apple
Push Notification service as an example.

In his talk on ``Collaborative intrusion handling using the
Blackboard-Pattern''~\cite{herold:iscs:2016}, Holger Kinkelin (TUM) presented an
approach how the individual components of intrusion handling (intrusion
detection, alert processing, intrusion response) can be better intertwined using
a blackboard as an information broker between components.

%Group discussion

Inspired by Hannes Tschofenig's talk on ``IoT Security'', a breakout session was
formed covering this topic. The participants collected their concerns about IoT
security, in particular collateral damage caused by vulnerable IoT devices in
the Internet, and listed challenges with IoT security and options for
mitigation. This discussion included, for instance, mandatory firmware updates
to close vulnerabilities from remote. This approach however is also problematic,
as the update might introduce incompatibilities with running applications or
might lead to data leaks harming user privacy \cite{Haus:CST2017}.

Another group discussed Blockchains and other types of distributed ledgers as
foundation for security and privacy solutions. Based upon a distributed ledger,
a trustworthy logging mechanisms could for instance be created that store
information from autonomous systems (cars, planes, etc.) for post-mortem
analyses in a non-mutable and non-modifiable manner. Another application example
includes a ledger-based configuration distribution mechanism for systems of any
kind, which could bring clear benefits namely accountability and transparency of
configuration.

The third group we want to highlight discussed on cloud security. Companies
own less physical hardware but lease  more and more virtual machines or services
in the Cloud. Besides amplifying known problems in traditional fields like
security, trust, verifiability, and privacy, the cloudification also brings
entirely new questions and problems. One example are services that already
utilize virtualization, for example, sandboxes that analyze malware. The nesting
of virtualization will decrease performance and change visibility of the malware
analyzer to the inspected malware.

Besides highlighted talks and group discussions, security and privacy topics
were also represented by following posters ``Securebox - A Platform for Securing
IoT Networks''~\cite{hafeez:s3:2015, hafeez:can:2016}, posters highlighting
ongoing research projects like ``SafeCloud''~\cite{safecloud},
``sKnock: Scalable Secure Port Knocking"~\cite{dsel:mcsp:2016},
``AutoMon''~\cite{automon}, and ``Towards an Information Model for Decentralized
Anomaly Detection for DecADe''~\cite{decade} and finally ``Research Directions
in Internet Architecture and Security''~\cite{glra, mms, ipv6hitlist}.
