%**************************************************************************
\section{Evolving the Retreat}\label{sec:evolution}
%**************************************************************************

The last action point on the MIRR agenda is collecting feedback of our participants. In this section we want to summarize most often suggested ideas for improvement, praise, and critics from the 2\textsuperscript{nd} MIRR.

The by far most often suggested improvement was to increase the amount of interactive elements in the agenda. Participants liked both poster and breakout sessions with a preference towards the breakouts as some persons proposed to have even two slots for breakout sessions.

Further interesting feedback was that participants would have liked to have the opportunity to prepare for the retreat in advance. For preparation they suggested to collect, publish, and refine topics of breakout sessions days or weeks ahead of the actual event, for instance, via mailing groups or more modern online discussion systems. Furthermore, a leader of the breakout session shall be elected in advance, how presents the topic to the plenum before the breakout session starts.

The just descibed online interaction might be even beneficial as it could help to ``break the ice'' between participants who are (quoting here) ``shy IT guys who mostly have not met before''.

Making introduction slides of participants, abstracts of talks and posters available before the event was also proposed by some participants. This step might prove helpful to cherry pick partners for discussions.

Both, representatives of academia and industry, stated that the participation of industry was too low. This is true for the number of persons (so far the academia to industry split was about 80\% to 20\%) but also for active contributions of representatives of companies, i.e. talks or topics for breakout sessions. Participants suggested that short industry talks that describe problems or open questions could be a good starting point for breakout sessions.

Participants enjoyed also the talks but felt that shorter talks of max. ten minutes would be enough to spark further discussions in smaller groups after the talk. Spending less time for ``compulsory'' talk sessions might also mitigate a potential weakness of the MIRR format pointed out by a participant: the participant felt that the scope of the retreat was too wide. While we understand the problem, we also see the advantage of getting to know work of other fields of network and Internet research as this can help to break out of ones own ``mental filter bubble'' and broaden ones overall knowledge. However, the amount of time for non-relevant content spent by an individual participant can be decreased when time is shifted from compulsory agenda points to such ones that can be picked by the participant individually.

Feedback by another person clashes somehow with above observation: the participant proposed to include talks in tutorial style. While some agreed, other participants pointed out that such talks would spend some time not available anymore for group work. This finding led to the proposal of having parallel tracks for talks.
