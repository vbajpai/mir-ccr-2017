%**************************************************************************
\section{Evolving the Retreat and Lessons Learned} \label{sec:evolution}
%**************************************************************************

The last action point on the MIRR agenda is collecting feedback from our participants. In this section we want to summarize most often suggested ideas for improvement, praise, but also critics from the 2\textsuperscript{nd} \ac{MIRR}.

The most surprising but also most controversially discussed feedback was that some participants would have liked to have the opportunity to prepare for the retreat in advance. Others responded that they would not have time for this.
The preparation supporters especially suggested to collect, publish, and refine topics of breakout sessions several days or a few weeks ahead of the actual event. For this purpose, one participant proposed to use mailing lists or more modern online discussion systems. The just described online interaction might even prove beneficial to ``break the ice'' between participants, who are (quoting another participant) ``typically quite shy IT guys who mostly have not met before''.
Furthermore, a leader of the breakout session shall be elected in advance, who quickly presents the topic to the plenum before the breakout session starts. Hence, participants who had no time to prepare for the retreat can still select the group of their choice group easily.
Further ideas were to make introduction slides of participants, abstracts of talks and posters available before the event. This step might prove helpful to cherry pick partners for discussions.
In our opinion, an approach that allows participants to prepare for the event but does not leave others behind is sound. Hence we will implement preparation in the upcoming 3\textsuperscript{rd} \ac{MIRR}.

Both, representatives of academia and industry, stated that the participation of industry was too low. This is true for the number of persons (so far the academia to industry split was about 80\% to 20\%) as well as for active contributions of representatives of companies, i.e. talks or topics for breakout sessions. Participants suggested that short industry talks that describe problems or open questions could be a good starting point for breakout sessions or provide interesting research ideas to the academia.

The by far most often suggested improvement was to increase the amount of interactive and individual elements in the agenda. Participants liked both poster and breakout sessions with a slight preference towards the breakouts. Some persons even proposed to have two slots for breakout sessions. Participants mostly enjoyed the talks given in the plenum but felt that shorter talks of max. ten minutes would be enough to spark further discussions in smaller groups after the talk. Spending less time for ``compulsory'' talks might also mitigate a potential weakness of the MIRR format pointed out by a participant: the participant felt that the scope of topics of the retreat was too wide. While we understand the problem, we also see the advantage of getting to know work of other fields of network and Internet research as this can help to break out of ones own ``mental filter bubble'' and broaden ones overall knowledge. However, the amount of time one spends with (personally experienced) non-relevant content can be decreased when time is shifted from compulsory agenda points to such ones that can be picked by the participant individually. This step is, as we believe, highly important to optimize a participants personal outcomes from and perception of the retreat. Hence, shorter compulsory talks to maximize the amount of individual interaction is another important aim for the upcoming \ac{MIRR}.

One participant proposed to include talks in tutorial style. While some agreed, other participants pointed out that such talks would again reduce the time available for more interactive action points. This finding led to the proposal of having two parallel tracks for talks.

Further feedback included to increase gender diversity and have changing, random seating arrangements during meal-times. Even though we think gender diversity is desirable, we also feel that this suggestion is difficult to implement as our quite technical field seems to attract mostly men. Random seating arrangements can help to mixing people but could also backfire when participants feel patronized as the cannot talk to who they want.
