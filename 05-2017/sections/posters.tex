%**************************************************************************
\section{Posters}\label{sec:posters}
%**************************************************************************

Participants were encouraged to bring posters to present
their recent research work.

%-- Hagen Paul Pfeifer
\subsection{Dynamic MultiPath Routing}


%-- Volker Hilt
\subsection{Edge Clouds - Challenges and Solutions}

%-- Michio Honda
\subsection{PASTE: A Networking Interface for NVMMs}

Emerging Non-Volatile Main Memories (NVMMs), also known as storage-class
memory and persistent memory  push the majority of end-to-end latency that
includes durable I/O to network stacks and their APIs.  This not only impairs
inherent performance of NVMMs that is one to two orders of magnitude faster
than traditional persistent medias like SSDs, but prevents systems from
adopting them to be reliable with relative ease. Michio Honda (NEC) presented
an investigation of this problem, designing an efficient network
stack~\cite{mhonda:hotnets:2016} and its APIs, and exploring new opportunities
in networking such as software switches and middleboxes in addition to
improving networked storage systems.

%-- Ermias Walelgne
\subsection{Measuring the Performance of Mobile Users}

In a mobile network, the mobile terminal (MT) continuously exchanges link
related metrics and signals to the nearby base station to measure the strength
and quality of the received signal. \ac{QoS} metrics are used for handover
decisions and cell reselection.  A handover can occur if there is a strong
radio signal in the neighboring cell while the serving cell’s radio signal is
getting diminished.  However, previous studies show that it is not always the
value of signal strength that matters to have a good throughput performance.
Therefore, knowing the possible achievable throughput value before making a
handover is equally important along with link-related \ac{QoS} metrics. Ermias
Walelgne (Aalto University) proposes a solution to estimate the throughput
value of post-handover using the metrics collected from the current serving
base station. The result of this throughput prediction can be combined with
other link QoS metrics such as RSSI and RSPQ values for better handover
decision.


%-- Roberto Morabito
\subsection{Lightweight Virtualization for Smart Cars}

Modern vehicles are equipped with several interconnected sensors on board for
monitoring and diagnosis purposes; their availability is a main driver for the
development of novel applications in the smart vehicle domain. Roberto
Morabito~\cite{rmorabito:im:2017} presented a Docker container-based platform
as solution for implementing customized smart car applications. Through a
proof-of-concept prototype—developed on a Raspberry Pi3 board—we show that a
container-based virtualization approach is not only viable but also effective
and flexible in the management of several parallel processes running on
On-Board Unit. More specifically, the platform can take priority-based
decisions by handling multiple inputs, e.g., data from the CANbus based on the
OBD II codes, video from the on-board webcam, and so on. Results are promising
for the development of future in-vehicle virtualized platforms.

%-- Ljubica Pajević Kärkkäinen
\subsection{Data-driven Mobility Modeling}

%-- Michael Haus
\subsection{iConfig - What I See is What I Configure}

Michael Haus (TU Munich) presented iConfig to manage \ac{IoT} devices in smart
cities. The management of \ac{IoT} devices in urban areas is becoming
important due to that the majority of the people living in cities and the
number of deployed \ac{IoT} devices are increasing. Therefore, iConfig
addresses three major issues in current \ac{IoT} management: registration,
configuration, and device maintenance. To achieve the goals of iConfig, the
presented system relies on programmable edge modules, which can run on
smartphones, wearables, and smart boards to configure physically proximate
\ac{IoT} devices.


%-- Teemu Kärkkäinen
\subsection{Opportunistic Content Dissemination}


%-- Daniel Herzog
\subsection{Recommender Systems \& Mobility Services}

\ac{RSs} in tourism often recommend single \ac{POIs} such as restaurants or
museums. However, tourists visiting a destination are usually looking for a
tourist trip composed of multiple \ac{POIs} along a practical route. Daniel
Herzog (TU Munich) presented a \ac{RS}~\cite{dherzog:it:2017} recommending
tourist trips to a group of users.  This is a particularly complex problem as
the \ac{RS} has to aggregate the travel preferences of all group members
before wgenerating recommendations.  Furthermore, we want to research how
different devices and user interfaces can support groups in providing feedback
on recommendations and finding a consensus.

%-- Lars Wischhof
\subsection{Data Dissemination in Vehicular Networks}

Lars Wischhof (Hochschule Müncher) presented an architecture and preliminary
results of an on-going research project at the research group where
communication schemes combining cellular communication with
direct-communication (such as \ac{D2D} modes of the latest LTE-A releases or
LTE-V) are combined for applications in intelligent mobility. The basic
assumption is that future vehicles will most-likely have multiple
communication technologies and modes available. Therefore, a context-aware
selection of the communication mode is advocated. A suitable architecture is
outlined. First simulation results for the example of a DENM-based application
indicate that a context-aware selection can outperform a static assignment.


%-- Severin Kacianka
\subsection{Accountability for Cyber-Physical Systems}

Severin Kacianka (TU Munich) seeks to capture the essential features of an
accountable (computer-)system.  Logs are, for example, a common way to create
evidence and establish  "truth" in computer systems. Another facet are
mechanisms to process those logs and techniques to formulate the questions of
compliance with laws as queries against those logs.  However, there are
currently no "blue prints" on how to make a system "accountable". We wish to
develop a comprehensive framework that makes it possible to explicate the
accountability features of a system, reason about their effectiveness, compare
it to other solutions and offer options to exchange one specific component for
another.


%-- Edwin Cordeiro
\subsection{Real-time TE in the Internet}


%-- Vittorio Cozzolino
\subsection{Fine-Grained Edge Offloading for IoT}

Vittorio Cozzolino (TU Munich) makes the case for \ac{IoT} edge offloading,
which strives to exploit the resources on edge computing devices by offloading
fine-grained computation tasks from the cloud closer to the users and data
generators (i.e., IoT devices). The key motive is to enhance performance,
security and privacy for IoT services. The proposal bridges the gap between
cloud computing and IoT by applying a divide and conquer approach over the
multi-level (cloud, edge and IoT) information pipeline.  To validate the
design of IoT edge offloading, a unikernel-based prototype is developed and
evaluated the system under various hardware and network conditions.
