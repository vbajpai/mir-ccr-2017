%**************************************************************************
\section{Invited Presentations}\label{sec:invited-presentations}
%**************************************************************************

The invited presentations were intended as a basis for triggering discussions
and identifying areas for group work.

%-- Teemu Kärkkäinen
\subsection{Opportunistic Content Dissemination Performance in Dense Network Segments}

Many of the existing opportunistic networking systems have been designed
assuming a small number links per node and have trouble scaling to large
numbers of potential concurrent communication partners. In the real world we
often find wireless local area networks with large numbers of connected users
– in particular in open Wi-Fi networks provided by cities, airports,
conferences and other venues. In this talk, Teemu Kärkkäinen (TU Munich)
presented a 50 client opportunistic network in a single Wi-Fi access point and
use it to uncover scaling problems and to suggest mechanisms to improve the
performance of single segment dissemination. Further, we present an algorithm
for breaking down a single dense segment dissemination problem into multiple
smaller but identical problems by exploiting resource (e.g., Wi-Fi channel)
diversity, and validate our approach via simulations and testbed experiments.
The ability to scale to high density network segments creates new, realistic
use cases for opportunistic networking applications.

%-- Quirin Scheitle
\subsection{Precise User Tracking Based on TLS Client Certificate Authentication} %(TMA'17) -- can scale to 15 to 25 min —

The design and implementation of cryptographic systems offer many subtle
pitfalls. One such pitfall is that cryptography may create unique identifiers
potentially usable to repeatedly and precisely re-identify and hence track
users. Quirin Scheitle (TU Munich) presented his investigation of TLS Client
Certificate Authentication (CCA), which currently transmits certificates in
plain text. He demonstrated~\cite{qscheitle:tma:2017} CCA’s impact on client
traceability using Apple’s Apple Push Notification service (APNs) as an
example. APNs is used by all Apple products, employs plain-text CCA, and aims
to be constantly connected to its backend. Its novel combination of large
device count, constant connections, device proximity to users and unique
client certificates provides for precise client traceability. He shows that
passive eavesdropping allows to precisely re-identify and track users and that
only ten interception points are required to track more than 80 percent of
APNs users due to global routing characteristics. The work was conducted under
strong ethical guidelines, with responsibly disclosing the findings, and a
working patch by Apple for the highlighted issue was confirmed. The aim for
this work is to provide the necessary factual and quantified evidence about
negative implications of plain-text CCA to boost deployment of encrypted CCA
as in TLS 1.3.

%-- Dirk Kutscher
%\subsection{Tentatively: IoT Research Ideas}
%$--$ Dirk Kutscher

%-- Hagen Paul Pfeifer
\subsection{Dynamic MultiPath Routing Protocol}

Safety critical communication platforms often deploy multiple heterogeneous
wireless link layer access technologies like ETSI TETRA, IEEE 802.11, IEEE
802.15.4, 3GPP LTE, satellite terminals or proprietary waveforms.  Depending
on the interface characteristics, vendor specific decisions and other aspects,
these often come shipped with suitable mobile ad-hoc routing protocols to form
networks in an autonomous manner - independently for each link type. 802.11
links typically deploy OLSR, BATMAN or 802.11s, whereas proprietary waveforms
often come with proprietary MANET protocols.  To bridge these access networks
at a logically higher level and provide an opaque operational network view, a
network routing protocol at the top is required. Exterior routing protocols
like BGP are limited in their use and features like automatic neighbor
detection or reduced message overhead are required. \ac{DMPR} tries to address
these shortcomings and provides exterior routing protocol features for
heterogeneous link layer environments even in low bandwidth environments.
Furthermore, \ac{DMPR} features policy based routing to route traffic through
different paths if required or advantageous.

%-- Hannes Tschofenig
\subsection{Internet of Things Security: TrustZone for v8-M architecture}

Internet of Things (IoT) devices today use microcontrollers that are limited
in CPU performance, as well as in RAM and flash size. Many of these devices
use the ARM A-class or M-class processors. A-class CPUs are able to run
popular operating systems (OS), such as embedded Linux, while M-class CPUs use
Real-Time Operating Systems. So far, the security functionality of M- and
A-class CPUs has been very different.  A-class CPUs can use sophisticated
hardware features, such as TrustZone offering physical separation between the
normal and the secure world operating systems. M-class CPUs offer basic
security protection using the memory protection unit (MPU), which offers
memory isolation.  Many IoT products, however, use very few operating system
security techniques, if they run an OS at all, and often do not make use of
hardware security support. While the exact reasons for these design decisions
remain unclear, the growing list of IoT security failures calls for improved
protection capabilities.  With the introduction of the TrustZone for ARMv8-M
architecture, security features known from the mobile world are now available
in the IoT environment. Additionally, ARM v8-A processors  enhance security
with the Pointer Authentication Extension, which prevents return-oriented
programming-based attacks.  In this talk Hannes will explain the hardware
security features offered by upcoming ARM processors and microcontrollers.

%-- Michio Honda
%\subsection{Persistence in Networking: Redesigning Stack, API and Networks}
\subsection{Redesigning Stack, API and Networks}

Emerging Non-Volatile Main Memories (NVMMs), also known as storage-class
memory and persistent memory  push the majority of end-to-end latency that
includes durable I/O to network stacks and their APIs.  This not only impairs
inherent performance of NVMMs that is one to two orders of magnitude faster
than traditional persistent medias like SSDs, but prevents systems from
adopting them to be reliable with relative ease.  Our work investigates
solving this problem, designing an efficient network stack and its APIs, and
exploring new opportunities in networking such as software switches and
middleboxes in addition to improving networked storage systems.
