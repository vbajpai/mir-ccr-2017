%**************************************************************************
\section{Invited Presentations}\label{sec:invited-presentations}
%**************************************************************************

The invited presentations were intended as a basis for triggering discussions
and identifying areas for group work.

%-- Teemu Kärkkäinen
\subsection{Opportunistic Content Dissemination Performance in Dense Network Segments}

Many of the existing opportunistic networking systems have been designed
assuming a small number links per node and have trouble scaling to large
numbers of potential concurrent communication partners. In the real world we
often find wireless local area networks with large numbers of connected users
– in particular in open Wi-Fi networks provided by cities, airports,
conferences and other venues. In this talk, Teemu Kärkkäinen (TU Munich)
presented a 50 client opportunistic network in a single Wi-Fi access point and
use it to uncover scaling problems and to suggest mechanisms to improve the
performance of single segment dissemination. Further, we present an algorithm
for breaking down a single dense segment dissemination problem into multiple
smaller but identical problems by exploiting resource (e.g., Wi-Fi channel)
diversity, and validate our approach via simulations and testbed experiments.
The ability to scale to high density network segments creates new, realistic
use cases for opportunistic networking applications.

%-- Quirin Scheitle
\subsection{Precise User Tracking Based on TLS Client Certificate Authentication} %(TMA'17) -- can scale to 15 to 25 min —

The design and implementation of cryptographic systems offer many subtle
pitfalls. One such pitfall is that cryptography may create unique identifiers
potentially usable to repeatedly and precisely re-identify and hence track
users. Quirin Scheitle (TU Munich) presented his investigation of TLS Client
Certificate Authentication (CCA), which currently transmits certificates in
plain text. He demonstrated~\cite{qscheitle:tma:2017} CCA’s impact on client
traceability using Apple’s Apple Push Notification service (APNs) as an
example. APNs is used by all Apple products, employs plain-text CCA, and aims
to be constantly connected to its backend. Its novel combination of large
device count, constant connections, device proximity to users and unique
client certificates provides for precise client traceability. He shows that
passive eavesdropping allows to precisely re-identify and track users and that
only ten interception points are required to track more than 80 percent of
APNs users due to global routing characteristics. The work was conducted under
strong ethical guidelines, with responsibly disclosing the findings, and a
working patch by Apple for the highlighted issue was confirmed. The aim for
this work is to provide the necessary factual and quantified evidence about
negative implications of plain-text CCA to boost deployment of encrypted CCA
as in TLS 1.3.

%-- Dirk Kutscher
\subsection{Tentatively: IoT Research Ideas}

%-- Hagen Paul Pfeifer
\subsection{Dynamic MultiPath Routing Protocol}

%-- Hannes Tschofenig
\subsection{Internet of Things Security: TrustZone for v8-M architecture}

