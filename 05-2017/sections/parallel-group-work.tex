%**************************************************************************
\section{Parallel Group Work}\label{sec:parallel-group-work}
%**************************************************************************

The afternoon sessions were used to discuss selected topics in more depth in
smaller groups. This section summarizes the discussions of each group.

% -------------

%-- Holger Kinkelin
%-- Severin Kacianka
\subsection{distributed ledgers + distributed Internet + empowering people}

%-- Quirin and Vaibhav
\subsection{Measurements and Reproducibility}

% LPK: \emph{Participants:} \textbf{Vaibhav Bajpai, Georg Carle, Edwin
% Cordeiro, Ljubica Pajević Kärkkäinen, Simon Leinen, Johannes Naab, Stefan
% Neumeier, Quirin Scheitle, Ermias Walelgne}

%In the present day, however, we are building systems and technology that
%permit human everyday lives and therefore, inherently are dealing with
%various, sensitive data about people and technical aspects of their privacy.

%This framework allows implementation of a number of privacy-preserving
%protocols for aggregating network statistics.  LPK: {From slides: "Also, some
%notable datasets got de-anonymized". Citation for this?}

%\subsubsection{Publishing datasets and privacy issues}
%\subsubsection{Methods reproducibility}
%\subsubsection{Incentives for more open research}
%\subsubsection{Possible solutions for the lack of reproducibility}
%\subsubsection{Learning from practice in other fields}

Computer science and engineering community has traditionally been less
oriented towards providing reproducible
research~\cite{vbajpai:reproducibility:2017, qscheitle:reproducibility:2017}
and more about quickly publishing novel and promising results. In this group,
we discussed the current state and possible directions for improving the state
of reproducibility.

The discussion began with questions regarding user privacy and sharing
sensitive (network traffic, user identities, location, et al.) data since both
legal and ethical concerns often inhibit data sharing. The first step that
must be taken is data anonymization, which is difficult, sometimes
computationally demanding and furthermore, it may only work for a specific
research question. Once the data is sufficiently cleared of sensitive
information and prepared for publication, long-term planning for storing large
amounts of data is necessary.  Additionally, even when there is willingness to
take the necessary steps and enrich the research community by providing
collected data, the risk of a certain party's intentional de-anonymization
always remains. Actions that can be taken to enable more reproducible research
were also discussed. For instance, Burkhart~\cite{burkhart2011thesis} explores
cryptographic alternatives to data anonymization. Specifically, he revisits
applying secure multiparty computation (MPC) to the problem of aggregating
network data from multiple domains and develops a new framework with MPC
operations for processing high volume data in near real-time.

Without the detailed methodology, much of a reproduction efforts remain
guesswork. A detailed description is essential since it serves to validate
others' work; not providing enough information about the methods used, for
example regarding data processing, can cast a serious doubt on the methodology
and any of the results.  Provision of all the steps of the applied methodology
also assists other practitioners and enables faster progress of research.

From the author's perspective, there is a disbalance between the efforts to
prepare reproducible results and the potential reward. Since publications with
novel measurement data and/or methodologies are likely to get more citations,
for many authors the citation count is seen as a strong enough incentive.
Reproducers find it even less rewarding since replicating others' results is
time-consuming and receives little recognition. Although many conference call
for papers express the need by explicitly asking for papers that reproduce
previous work, there is currently little acceptance of reproduction papers:
most of the studies -- if ever materialize into a submission -- get filtered
out in the peer review round (due to the lack of novelty). Furthermore,
rebuttal papers on previously published results are often subjected to
rejection.

Instead of using very specific datasets for examining algorithm and protocol
performance, having benchmark datasets could be one part of the solution.
However, in this case the question that emerged was how we actually build
suitable datasets for different applications and use-cases.  The other action
item would be to promote awareness that the current practices do not
completely follow the scientific approach. This may be achieved by organizing
more workshops with the focus primarily on reproducible research and
validation of previous results, or by tweaking the reviewing process to
encourage more openness.

We also discussed what can we learn from other fields where validation of
results through replication has been an essential component. One of the
examples with very high requirements for validation is in physics, where
numerous teams work simultaneously on the same dataset and evaluate core
models from many angles. However, this field deals with zero privacy and human
subjects issues~\cite{fbai:infocom:2003}.

%-- Dominik Scholz
\subsection{P4 and SDN}

P4~\cite{pbosshart:ccr:2014} is a language that allows to program the
structural layout of protocol headers, as well as, processing operations
performed on those.  This allows to perform matching on arbitrary fields of
even completely new protocols, in contrast to for instance OpenFlow with few
predefined fields.  The breakout session aims to first define the
functionality and potential use cases of P4.  Then, potential
extensions~\cite{abhashkumar:sosr:2017} for functionality that might be
desired, but are not yet supported or even intended by P4 will covered.

%-- Hannes
\subsection{IoT and security}

Firmware updates~\cite{philips, hp} changing behaviour of the device.

%-- Lars Wischoff
\subsection{Networking APIs}
%\subsection{regulatory issues} -- merged

The first phase in the discussion of the group was to define the assumptions
under which the flaws and possible improvements of the Networking APIs, in
particular the standard Socket API, should take place.  These were the
following two basic assumptions:

\begin{itemize}

\item The \textbf{currently deployed Internet architecture} is assumed.  New
networking paradigms that would require completely new functionalities of the
Networking-API - are out-of-scope, e.g.: information centric / content based
networking, delay-tolerant networking, vehicular networking.

\item Specific requirements of \textbf{IoT are not considered} in the
discussion since the IoT devices have very specific requirements, i.e.
regarding energy-efficiency, so that the Socket API is often not applicable/is
not being used. If Socket API is used in an IoT device, energy-consumption is
not a problem since sending consumes much more energy than an inefficient
implementation of the Socket API (e.g.  copying data from userland to
kernelspace).

\end{itemize}

Afterwards, the discussion focussed on problems observed with the current
APIs. For high performance networking applications with, e.g., thousands of
TCP connections, the API does not scale well. Several work-arounds have been
developed in order to cope with this inefficiency, e.g.  \emph{sendfile}
(directly copying from file descriptor to a socket without copying data to
user-space) or  \emph{sendmessage} with zero-copying / zero-copy sockets (page
re-mapping between kernel/user-space).  The importance of hardware-offloading
has increased. However, several open questions remain: What functions should
be performed in hardware?  Should the interface to the hardware be
standardized?

The current API is fine for most standard cases but it becomes a bottleneck
for high performance applications: This is mainly caused by the effort of
copying data, often caused by a not packet-oriented processing of data in the
application (i.e. application sends stream of data, transport layer builds
segments).  Again, workarounds have been developed, e.g. fast packet
processing in userland or StackMap~\cite{mhonda:usenix:2016} + netmap
framework~\cite{lrizzo:usenix:2012} (dedicated NIC for one application, etc.).

This lead to a discussion of desirable properties of a networking API.
Mentioned were the isolation of networking-stack and application, energy
efficiency (mobile applications) and high performance and scalability (data
center applications).

Possible solution ideas which were identified during the breakout were the
application of dedicated I/O CPUs, integration of GPGPU processing and
networking (offloading on GPU), packetized processing of data in the
application, and several techniques that could reduce the processing-overhead
in kernel, e.g. avoid queuing of TCP ACK packets or reducing the overhead of a
system call. Due to the limited time, a detailed discussion of the suitability
of theses approaches would need to be done in a follow-up discussion.
Furthermore, two talks on the second day of the retreat presented additional
information on related subjects: Michio Honda (see~\autoref{mhonda}) presented
information on persistence in networking (redesigning stack, API and networks)
and Florian Westphal (see~\autoref{fwestphal}) talked about current
developments regarding the Linux networking stack.
