%**************************************************************************
\section{Parallel Group Work}\label{sec:parallel-group-work}
%**************************************************************************

The afternoon sessions were used to discuss selected topics in more depth in
smaller groups. This section summarizes the discussions of each group.

% -------------

%-- Holger Kinkelin
%-- Severin Kacianka
\subsection{distributed ledgers + distributed Internet + empowering people}

%-- Quirin and Vaibhav
\subsection{measurements and reproducibility}

\cite{fbai:infocom:2003}

%-- Dominik Scholz
\subsection{P4 and SDN}

P4~\cite{pbosshart:ccr:2014} is a language that allows to program the
structural layout of protocol headers, as well as, processing operations
performed on those.  This allows to perform matching on arbitrary fields of
even completely new protocols, in contrast to for instance OpenFlow with few
predefined fields.  The breakout session aims to first define the
functionality and potential use cases of P4.  Then, potential
extensions~\cite{abhashkumar:sosr:2017} for functionality that might be
desired, but are not yet supported or even intended by P4 will covered.

%-- Hannes
\subsection{IoT and security}

Firmware updates~\cite{philips, hp} changing behaviour of the device.

%-- Lars Wischoff
\subsection{networking APIs}
%\subsection{regulatory issues} -- merged

The first phase in the discussion of the group was to define the assumptions
under which the flaws and possible improvements of the Networking APIs, in
particular the standard Socket API, should take place.  These were the
following two basic assumptions:

\begin{itemize}

\item The \textbf{currently deployed Internet architecture} is assumed.  New
networking paradigms that would require completely new functionalities of the
Networking-API - are out-of-scope, e.g.: information centric / content based
networking, delay-tolerant networking, vehicular networking.

\item Specific requirements of \textbf{IoT are not considered} in the
discussion since the IoT devices have very specific requirements, i.e.
regarding energy-efficiency, so that the Socket API is often not applicable/is
not being used. If Socket API is used in an IoT device, energy-consumption is
not a problem since sending consumes much more energy than an inefficient
implementation of the Socket API (e.g.  copying data from userland to
kernelspace).

\end{itemize}

Afterwards, the discussion focussed on problems observed with the current
APIs. For high performance networking applications with, e.g., thousands of
TCP connections, the API does not scale well. Several work-arounds have been
developed in order to cope with this inefficiency, e.g.  \emph{sendfile}
(directly copying from file descriptor to a socket without copying data to
user-space) or  \emph{sendmessage} with zero-copying / zero-copy sockets (page
re-mapping between kernel/user-space).  The importance of hardware-offloading
has increased. However, several open questions remain: What functions should
be performed in hardware?  Should the interface to the hardware be
standardized?

The current API is fine for most standard cases but it becomes a bottleneck
for high performance applications: This is mainly caused by the effort of
copying data, often caused by a not packet-oriented processing of data in the
application (i.e. application sends stream of data, transport layer builds
segments).  Again, workarounds have been developed, e.g. fast packet
processing in userland or StackMap~\cite{mhonda:usenix:2016} + netmap
framework~\cite{lrizzo:usenix:2012} (dedicated NIC for one application, etc.).

This lead to a discussion of desirable properties of a networking API.
Mentioned were the isolation of networking-stack and application, energy
efficiency (mobile applications) and high performance and scalability (data
center applications).

Possible solution ideas which were identified during the breakout were the
application of dedicated I/O CPUs, integration of GPGPU processing and
networking (offloading on GPU), packetized processing of data in the
application, and several techniques that could reduce the processing-overhead
in kernel, e.g. avoid queuing of TCP ACK packets or reducing the overhead of a
system call. Due to the limited time, a detailed discussion of the suitability
of theses approaches would need to be done in a follow-up discussion.
Furthermore, two talks on the second day of the retreat presented additional
information on related subjects: Michio Honda (see~\autoref{mhonda}) presented
information on persistence in networking (redesigning stack, API and networks)
and Florian Westphal (see~\autoref{fwestphal}) talked about current
developments regarding the Linux networking stack.
