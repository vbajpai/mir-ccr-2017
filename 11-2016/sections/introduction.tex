%**************************************************************************
\section{Introduction}\label{sec:introduction}
%**************************************************************************

%------------------------ Motivation
%------------------------ Goals

The \ac{MIR} originated from informal discussions of different research groups
at TUM and a team at NetApp on diverse topics related to networking. The
discussions brought together PhD students and post-docs to present their
respective research (including both work in progress as well as polished
results) and provided an informal setting for intense and rich exchange among
participants involved.  We realised that there was notable potential in
reaching out further, which eventually led to the instantiation of the
\ac{MIR}.

The main mission of the \ac{MIR} is to ensure mutual awareness of different
teams working on current (complementary) topics in networking. We want to lay
the foundations for establishing, broadening, and deepening cooperation among
a variety of groups doing networking research. In order to foster easily
sustainable relationships, our initial scope has been deliberately limited to
the area around Munich (which may reach as far as 400 km in some cases).  As a
common denominator, we target like-minded teams within the region, where the
common mindset stems from practical research in networked systems, paired with
interest and efforts in the \ac{IETF}, the \ac{IRTF} and the ACM SIGCOMM and
SIGMOBILE communities.

The purpose of the \ac{MIR} is threefold: 1) We seek to provide recurring
opportunities for companies to get in touch with research groups that have
expertise in fields relevant to the former.  2) We aim to support researchers
in understanding current and emerging research and engineering problems from
the commercial development and deployment perspectives.  3) We like to offer
reality feedback to academic researchers and out-of-the-box ideas to those
from industry.  Overall, we hope to foster future bi- or multi-lateral
collaboration between academics and industry.

The retreat is organized in a highly interactive fashion, combining posters
(for providing variety) and group discussions intertwined with plenary talks
that stimulate discussions.  Organization directions are shaped by the
feedback of the participants, keeping the format constantly evolving.  We
borrow some elements from the renowned Dagstuhl seminars: We limit the number
of participants to $\sim$40 to maintain interactivity and allow all
participants to meet one another.  We hold the retreat in Raitenhaslach away
from the daily activities to ensure focus and include an overnight stay and a
social dinner to foster continued interaction and allow for digesting ideas.
The seminar is by invitation only, and we put an emphasis on the industry,
picking PhD students with matching topics, which helps with obtaining a
compatible and energetic mix of people. Because we know that everybody's time
is scarce, we organize each retreat in a way that it occupies just two days
including arrival and departure.  With a target of two workshops per year,
presently scheduled for May and November, we shall be able to continuously
engage with a growing regional community even if individuals cannot
participate on every occasion.

Towards this mission, the 1\textsuperscript{st} \ac{MIR} retreat was organized
on November 24--25, 2016 at the TU Munich (TUM) Science and Study Center in
Raitenhaslach, Germany. Presentations on topics such as: \ac{SDN}, \ac{NFV},
\ac{ICN}, \ac{IoT}, Internet measurements and security-related research were
solicited.  The retreat consisted of ten invited presentations and several
posters presenting early and upcoming research, with six breakout sessions to
discuss topics of interests in an informal setting. Synopses of these sessions
are described in this report in more detail.
